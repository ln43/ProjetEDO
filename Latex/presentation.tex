\documentclass[10pt]{beamer}

\usetheme[]{JuanLesPins}
\setbeamercolor{structure}{fg=brown}
\setbeamertemplate{navigation symbols}{}
\setbeamertemplate{footline}[frame number]
%-------------------------------------------------------
% INCLUDE PACKAGES
%-------------------------------------------------------

\usepackage[utf8]{inputenc}
\usepackage[francais]{babel}
\usepackage[T1]{fontenc}
\usepackage{helvet}
\usepackage{graphicx}
\usepackage{amsmath,amsfonts,amssymb}
\usepackage{enumitem}
\usepackage{float}
\usepackage{hyperref}
\usepackage{cases}

%-------------------------------------------------------
% INFORMATION IN THE TITLE PAGE
%-------------------------------------------------------

\title[Population Néanderthal]{\textbf{Evolution de la population d'Homo Néanderthal}}
\author[Y. Adimy M. Simon H. Vassal]{Y. Adimy M. Simon H. Vassal}
\institute[]{INSA Lyon - Bioinformatique et Modélisation}
\date{13 Juin 2016}
\titlegraphic{\includegraphics[width=3cm]{neanderthal-sapiens.jpg}}
\logo{\includegraphics[scale=0.1]{logo_insa.png}}

%-------------------------------------------------------
% THE BODY OF THE PRESENTATION
%-------------------------------------------------------

\begin{document}

%-------------------------------------------------------
% THE TITLEPAGE
%-------------------------------------------------------

\begin{frame}[plain,noframenumbering] 
   \titlepage
   \insertlogo
\end{frame}

\begin{frame}{Contexte biologique}{}
\end{frame}

\begin{frame}{Présentation des modèles étudiés}{Cadre général}
\begin{block}{Modèle général}
	\begin{equation}
		\frac{\partial u(t,x)}{\partial t}=f(u(t,x))+d\Delta u(t,x), \quad t \in \mathbb{R}, x \in \mathbb{R}
	\end{equation}
\end{block}
\begin{itemize}
	\item u(t,x) : Densité de population  $\ \in[0,1]$ 
    \item $d$ : Constante de diffusion 
\end{itemize}
\end{frame}

\begin{frame}{Présentation des modèles étudiés}{Croissance logistique}
\begin{block}{Croissance Logistique}
	$$\frac{\partial u(t,x)}{\partial t}=f(u(t,x))+d\Delta u(t,x)$$
	$$f(u(t,x))=\alpha u (1 - \dfrac{u}{K}) $$
\end{block}
\begin{itemize}
    \item $K$ : Capacité de transport 
    \item $\alpha$ : Taux de croissance maximum
\end{itemize}
\end{frame}

\begin{frame}{Présentation des modèles étudiés}{Modèle Allee}
\begin{block}{Modèle Allee}
	$$\frac{\partial u(t,x)}{\partial t}=f(u(t,x))+d\Delta u(t,x)$$
	$$f(u(t,x))=ku(1-u)(u-A)$$
\end{block}
\begin{itemize}
    \item $k$ : Taux de croissance normalisé constant 
    \item $A$ : Densité critique
\end{itemize}
$$k=\frac{4}{(1-A)^2}$$
\end{frame}

\begin{frame}{Présentation des modèles étudiés}{Système de Lolkta-Volterra}
\begin{block}{Modèle de compétition}
	$$\begin{cases} \frac{\partial u(t,x)}{\partial t} = f(u,v) + d_1\Delta u\\ \frac{\partial v(t,x)}{\partial t} = g(u,v) + d_2 \Delta v \\ 
\end{cases}$$
	$$f(u,v) = \alpha_1 u\left(1-\frac{u}{K_1}-\gamma_1\frac{v}{K_1}\right) \text{, } g(u,v) = \alpha_2 v\left(1-\frac{v}{K_2}-\gamma_2\frac{u}{K_2}\right)$$
\end{block}
\begin{itemize}
	\item u(t,x) : Densité de population des Hommes Modernes 
    \item v(t,x) : Densité de population des Hommes de Néanderthal 
    \item $K_1$ et $K_2$ : Capacités d'accueil du milieu
    \item $\gamma_1$ et $\gamma_2$ : Coefficients de compétition
    \item $\alpha_1$ et $\alpha_2$ : Taux de croissance
\end{itemize}
\end{frame}

\end{document}